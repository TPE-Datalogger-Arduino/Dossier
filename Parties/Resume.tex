\chapter*{Résumé \&{} remerciements}

Depuis quelques années maintenant, nous assistons à un réel développement de la domotique. De nombreux produits sont disponibles pour l'utilisateur afin de rendre sa maison plus \og intelligente \fg.

La \emph{domotique} est l'ensemble des techniques utilisés pour nous rendre le quotidien meilleur. Ce domaine utilise quasiment tout les techniques existantes : nous pouvons trouver aussi bien de l'électronique que de la physique ou encore de l'informatique. Ces dispositifs perment, entre autre, de contrôler notre maison depuis l'extérieur : fermeture et ouverture des volets, déclenchement du chauffage, sécuriser notre habitat, \dots Mais il peut également servir à augmenter notre comfort et embellir notre quotidien comme les home-cinéma, la musique multi-pièce, \dots

Aujourd'hui, un des rôles très important de la domotique est de réduire les coûts financiers et les impacts sur l'environnement. Ainsi, des objets technologiques comme des thermostats intelligents ou des capteurs de température se développent.

\Espace

Par la biais de nos travaux personnels encadrés, nous avons décidés de nous diriger vers ce domaine. Tout au long de ce dossier, nous nous questionnerons sur
\begin{quotation}
	\noindent\itshape Comment relever des données météorologiques et les transmettre sur un espace accessible par tous ?
\end{quotation}
Ainsi, pour combler cette problèmatique, nous nous sommes penchés sur le développement et la fabrication d'un releveur de données météorologiques qui est capable de transmettre ses données sur Internet. Notre dispositif permettra alors de contrôler la température et la pression d'une pièce dans le but de réguler le chauffage. Celui-ci peut aussi faire office de mini station météorologique qui deliverai, en temps réel sur le Web, la température et la pression.

Notre dossier s'organisera en plusieurs parties. Dans un premier temps, nous allons vous présenter notre projet avec son analyse fonctionnelle. Ensuite, nous expliquerons en détaiils notre démarche pour créer cet objet avec la partie électronique et site Web. Enfin, nous présenterons les résultats que nous aurons obtenus. À la fin de ce dossier, vous pourrez retrouver en annexe les codes complets utilisés.

\Espace

Nous pouvons remercier nos deux professeurs de sciences de l'ingénieur M.~\bsc{Guibert} et M.~\bsc{Labour\-dette} qui nous ont conseillés tout au long de nos recherches.