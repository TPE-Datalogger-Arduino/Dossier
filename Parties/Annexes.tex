\chapter{Code complet pour l'Arduino}
\label{code:arduino}

Note : Ce code est configuré pour fonctionner en réseau local.

\FichierCode{C++}{Codes/Arduino/Programme_complet.ino}

\chapter{Code complet du site Web}
\label{code:site}

Le code complet du site Web se trouve ci-dessous. Il est également téléchargeable sur notre dépôt GitHub :
\begin{center}
	\url{http://github.com/TPE-Datalogger-Arduino/Site}
\end{center}

\section{Fichiers d'inclusion}

\cprotect\subsubsection{\verb-includes/settings.php-}

C'est le fichier comportant les paramètres principaux du site comme le serveur de la base, le mot de passe, \dots
\FichierCode{PHP}{Codes/TPE-Site/includes/settings.php}

\cprotect\subsubsection{\verb-includes/connect.php-}

C'est le fichier qui permet de se connecter à la base.
\FichierCode{PHP}{Codes/TPE-Site/includes/connect.php}

\cprotect\subsubsection{\verb-includes/sql.php-}

C'est le fichier qui contient toutes les fonctions pour extraire les données de la base. (La fonction pour exporter en CSV n'a pas été expliquée mais elle est relativement simple.)
\FichierCode{HTML+PHP}{Codes/TPE-Site/includes/sql.php}

\cprotect\subsubsection{\verb-includes/head.php-}

C'est l'en-tête HTML de toutes les pages.
\FichierCode{HTML+PHP}{Codes/TPE-Site/includes/head.php}

\cprotect\subsubsection{\verb-includes/footer.php-}

De-même, le pied de page.
\FichierCode{HTML+PHP}{Codes/TPE-Site/includes/footer.php}

\section{Style et script pour le site}

\cprotect\subsubsection{\verb-statics/styles/main.css-}

C'est le fichier CSS principale régissant toute l'apparence du site.
\FichierCode{CSS}{Codes/TPE-Site/statics/styles/main.css}

\cprotect\subsubsection{\verb|statics/scripts/mobile-menu.js|}

C'est un petit fichier JavaScript permettant de créer un menu déroulant lorsqu'on accède au site depuis un \emph{smartphone}.
\FichierCode{JS}{Codes/TPE-Site/statics/scripts/mobile-menu.js}

\section{Pages pour ajouter ou visualiser les données}

\cprotect\subsubsection{\verb-arduino.php-}

C'est le fichier qu'appele Arduino pour ajouter les données.
\FichierCode{HTML+PHP}{Codes/TPE-Site/arduino.php}

\cprotect\subsubsection{\verb-index.php-}

C'est la page d'accueil. Elle affiche le dernier relevé.
\FichierCode{HTML+PHP}{Codes/TPE-Site/index.php}

\cprotect\subsubsection{\verb-donnees.php-}

C'est la page pour afficher le tableau de données.
\FichierCode{HTML+PHP}{Codes/TPE-Site/donnees.php}

\cprotect\subsubsection{\verb/exporter-csv.php/}

C'est le fichier qui permet d'exporter les données au format CSV.
\FichierCode{HTML+PHP}{Codes/TPE-Site/exporter-csv.php}

\cprotect\subsubsection{\verb-graphique.php-}

C'est la page qui affiche le graphique.
\FichierCode{HTML+PHP}{Codes/TPE-Site/graphique.php}

\section{Autres pages}

\cprotect\subsubsection{\verb-403.php- et \verb-404.php-}

Ce sont les pages d'erreurs.
\FichierCode{HTML+PHP}{Codes/TPE-Site/403.php}
\FichierCode{HTML+PHP}{Codes/TPE-Site/404.php}
