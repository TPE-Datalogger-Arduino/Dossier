\chapter*{Conclusion}

Nous avons donc réussi à faire fonctionner et à mettre en relation chacun des éléments composant notre système. Ceci montre que nous avons atteint les objectifs que nous nous étions fixés.

Cependant, nous avons subi divers échecs. Par exemple, notre capteur n'avait pas une bonne fidélité et nous donnait des résultats étranges compte tenu de notre connaissance \emph{a priori} de l'environnement. Nous avons alors décidé de changer de capteur. De même, notre programme Arduino nécessitait ---~et nécessite toujours~--- un ordinateur pour fonctionner et effaçait l'intérêt du système

Nous pouvons cependant toujours faire évoluer notre projet en ajoutant des capteurs pour obtenir d'autres données (par exemple, l'humidité ou la luminosité), en étanchéifiant l'Arduino pour lui permettre de résister au milieu extérieur, ou encore en reliant plusieurs Arduino au serveur pour relever les données météorologiques à des endroits différents. Nous pourrions de même effectuer les relevés avec des capteurs plus précis, bien que notre utilisation ne le requiert pas.
